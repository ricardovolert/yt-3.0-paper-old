The process of transforming data into understanding constitutes the vast
majority of time, energy, and intellectual effort spent during scientific
inquiry.  This is true across domains, whether data is the product of a
computational simulation, a telescope observation, the synthesis of sensors
distributed across the Earth, or a collection of images of the human brain.
Data, by themselves, do not reflect an understanding of the Universe or its
underlying physical properties; rather, they are recordings, or measurements,
of the state of systems as observed.  Even for computational simulations, such
as simulations of star formation in the galaxy, this is true: these simulations
encode information about a discretization of a model, rather than the model
itself.

Bridging the gap between this discretization and the physical understanding
requires accessing data, manipulating and interrogating this data, and then
applying to this data a sense of understanding.  Somehow, bits stored on a disk
must become, in our minds, a galaxy undergoing a starburst.

This process is both mediated and impeded by computational tools.  When those
tools align with our mental model of how data exists, they can allow us to work
more efficiently, asking questions of data and building sophisticated
scientific inquiry.  However, when they do not, they can cause frustration,
delays, and most worryingly, incorrect or misinterpreted results.  When viewing
this from the perspective of the landscape of inquiry, the most startling
realization is that the questions a computational tool enables
individuals to ask shapes the questions they think to ask.

In \cite{yt_method_paper}, the analysis platform \yt{} was described.  At the
time, \yt{} was focused on analyzing and visualizing the output of grid-based
adaptive mesh refinement hydrodynamic simulations; while these were used to
study many different physical phenomena, they all were laid out in roughly the
same way, in rectilinear meshes of data.  In this paper, we present the current
version of \yt{}, which enables identical scripts to analyze and visualize data
stored as rectilinear grids as before, but additionally particle or discrete
data, octree-based data, and data stored as unstructured meshes.  This has been
the result of a large-scale effort to rewrite the underlying machinery within
\yt{} for accessing data, indexing that data, and providing it in efficient
ways to higher-level routines, as discussed in Section Something.  While this
was underway, \yt{} has also been considerably reinstrumented with
metadata-aware array infrastructure (Section \ref{sec:units}),
the volume rendering infrastructure has been rewritten to be more user-friendly
and capable (Section \ref{sec:vr}), and support for non-Cartesian geometries has
been added (Section \ref{sec:noncartesian}).

The single biggest change to \yt{} since that paper was published has not been
technical in nature.  In the intervening years, a directed and intense
community-building effort has resulted in the contributions from over a hundred
different individuals, many of them early-stage researchers, and a thriving
community of both users and developers (Section \ref{sec:community}).  This is
the crowning achievement of development, as we have attempted to build \yt{}
into a tool that enables inquiry from a technical level as well as fosters a
supportive, friendly community of individuals engaged in self-directed inquiry.
