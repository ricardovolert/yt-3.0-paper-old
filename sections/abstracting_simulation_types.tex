\subsection{Chunking and Decomposition Strategies}

% chunking types: io, all, spatial

Reading data, particularly data that will not be utilized in a computation, can
incur susbtantial overhead, particularly if the data is spread over multiple
files on a networked filesystem, where metadata queries can dominate the cost
of IO.  \yt{} takes the approach of building a coarse-grained index based on
the discretization method of the data (particle, grid, octree, unstructured
mesh), combining this with datapoint-level indexing for selection processes.

To supplement this, methods in \yt{} that process data utilize a system of data
``chunking," whereby segments of data identified during coarse-grained indexing
are subdivided by one of a few different schemes and yielded to the iterating
function; these schemes can include a limited number of tuning parameters or
arguments.  These three chunking methods are \texttt{all}, \texttt{spatial} and
\texttt{io}.  The \texttt{all} method simply returns a single, one-dimensional
array, and the number of chunks is always exactly one; this enables both
non-parallel algorithms and simple access to small datasets.  \texttt{spatial}
chunking yields three-dimensional arrays.  For grid-based datasets,
these are the grids, while for particle and octree datasets they are
leaf-by-leaf collections of particles or mesh values.  Optionally, the
\texttt{spatial} chunking method can return ``ghost zones" around regions, for
computation of stencils.  The final type of chunking, \texttt{io}, is designed
to iterate over sets of data in a manner that is most conducive to pipelined
IO.  These will not always be load-balanced in size of the returned chunks,
however.  In some cases, \texttt{io} chunking may return one file at a time (in
the case of spreading items across many different files), while in others it
may be returning sub-components of a single file.  This chunking type is the
most common strategy for parallel-decomposition.

\subsection{Grid Analysis}

\subsection{Octree Analysis}

\subsection{SPH Analysis}

\subsection{Unstructured Mesh Analysis}

\subsection{Non-Cartesian Coordinates}


