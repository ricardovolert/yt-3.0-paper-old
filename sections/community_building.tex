% Things that need to be addressed here:
%
%  * How do we interact with the community of users
%  * How do we steward individuals to contribute and develop the code
%  * What is the governance procedure?
%  * How are priorities set (or how aren't they set)?

Choosing a software package for a particular purpose involves evaluating
several differentiating factors; these factors include the functionality of a
package, the performance of a package, the user-friendliness, and even the
ability of an individual to find help, engage with others and feel a sense of
participation.  \textbf{cite something here} The development, fostering and
design of the community around \yt{} is deemed to be both crucial to the
success or failure of \yt{}, and in many ways inseparable from its
functionality.

\subsection{Composition}

There are several rough categories of individuals engaged in development and
utilization of \yt{}.  As a result of its API-first design, there are few if
any individuals who use \yt{} that do not do so through the scripting
interface; this means that the vast (if not exclusive) majority of individuals
who interact with the functionality in \yt{} are doing so by writing their own
scripts, modules, and code, and arguably engaging in a value-added development
process of their own.  The majority of individuals using \yt{} at present are
in astronomy and astrophysics, typically fields of simulation, although there
is an increasing group of individuals from other domains that are participating
in development and using \yt{} for their own domain-specific problems.

Making the distinction somewhat more clearly, there are individuals who have
built their own scripts and utilized them as well as individuals who have
contributed changes or modules to the primary \yt{} codebase.  In addition,
there is an emerging set of projects that build on \yt{} as infrastructure to
conduct scientific analysis.  These developers are largely driven by their own
pragmatic scientific needs, and they constitute the majority of developers (by
number) that contribute to the code base.  The majority of these individuals
are early- to mid-career researchers, typically graduate students, postdocs,
and assistant professors.

In recent years, there has emerged a more coherent contingency of individuals
who participate in both pragmatically-focused development of modules and
functionality for their own benefit as well as modules or overall improvement
that is supplemental or even external to their own research agenda.  These
improvements include improvements to the unit handling, to the plotting code,
to infrastructure for loading disparate datasets, and so on.  At this time we
do not know of any individuals funded to work on \yt{} completely independent
of a scientific or scholarly goal.

The composition of the community, particularly with a mixture of timelines for
goal-setting and completion, can at times cause frustrations and difficulties.
For instance, the response to "Can this feature be implemented?" often includes
an invitation for the questioner to collaborate on developing that feature and
submitting it to the codebase.  Developing a schedule of releases is an act of
consensus building, both deciding what bugs are critical to fix in the timeline
of a release as well as building consensus on what features should be
considered blockers for a new release.  The intersection of this with academic
deadlines (for instance job application season) requires balance and care.

\subsection{Types of Tasks}

When evaluating the level of engagement, we consider a few different
classifications of tasks that are performed by individuals in the community,
and evaluate these based on how they flow into greater engagement.

\begin{itemize*}
  \item Filing issues
  \item Participating in mailing list discussions
  \item Issuing a pull request
  \item Writing documentation
  \item Participating in code review
  \item Drafting an enhancement proposal
  \item Closing bug reports
\end{itemize*}

While there are other activities that individuals can participate in, these are
the typical activities we see among participants in the community.  The order,
flowing from the first to the last, is the typical flow we see for an
individual coming to participate in the community.  The first step is typically
to file an issue or bug report (occasionally these are requests for new
features), followed by partipating in development-focused discussion on mailing
lists.  The next level of engagement typically involves the development of a
new piece of functionality, refinement of existing code, or issuing a fix for a
bug or issue.  These take the form of pull requests (described in greater
detail in Section \ref{sec:development}) that can be reviewed and added to the
code base.

The next level of engagement centers around tasks that are not fully-aligned
with pragmatic, code-driven scientific inquiry.  The development of
documentation is often viewed as orthogonal to the scientific process, and
typically requires an iterative wrriting process.  Participation in code
review, providing comments, feedback and suggestions to other authors, is
another somewhat orthogonal task; it doesn't necessarily directly benefit the
developer doing the reviewing (although it might) and it does not necessarily
result in academic rewards (citations, authorship, etc).  But, it does arise
from a pragmatic (ensuring code reliability) or altruistic (the public good of
the software) motivation, and is thus a deeper level of engagement.

The final two activities, drafting enhancement proposals and closing bug
reports, are the most engaged, and often the most removed from the academic
motivation structure.  Developing an enhancement proposal (see Section
\ref{sec:development}) for \yt{} means iterating with other developers on the
motivation behind and implementation of a large piece of functionality; it
requires both motivation to engage with the community and the patience to build
consensus amongst stakeholders.  Closing bug reports -- and the development
work associated with identifying, tracking and fixing bugs -- requires patience
and often repeated engagement with stakeholders.

\subsection{Engagement Metrics}

We include here plots of the level of engagement on mailing list discussions
and the citation count of the original method paper.

\subsection{Governance}


