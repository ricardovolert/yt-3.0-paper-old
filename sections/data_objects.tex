The basic principles by which \yt{} operates are built on the notion of
selecting data (through coarse and subsequent fine-grained indexing of data
sources such as files), accessing that data in a memory-efficient fashion, and
then processing that data into either a resultant set of quantitative data or a
visualization.

The mechanisms by which \yt{} can select data are typically spatial in nature,
although several non-spatial mechanisms focused on queries can be utilized as
well.  These objects which conduct selection are selectors, and are designed to
provide as small of an API as possible, to enable ease of development and
deployment of new selectors.

Selectors require defining several functions, with the option of defining
additional functions for optimization, that return true or false whether a
given point is or is not included in the selected region.  These functions
include selection of a rectilinear grid (or any point within that grid),
selection of a point with zero extent and selection of a point with a non-zero
spherical radius.

The base selector object utilizes these routines during a selection operation
to maximize the amount of code reused between particle, patch, and octree
selection of data.  These three types of data are selected through specific
routines designed to minimize the number of times that the selection function
must be called, as they can be quite expensive.

Selecting data from a grid is a two-step process.  The first step is
identifying which grids intersect a given data selector; this is done through a
sequence of bounding box intersection checks.  Within a given grid, the cells
which are intersected are identified.  This results in the selection routine
being called once for each grid object in the simulation and once for each cell
located within an intersecting grid.  This can be conducted hierarchically, but
due to implementation details around how the grid index is stored this is not
yet cost effective.

Selecting data from an octree-organized dataset utilizes a recursive scheme
that selects individual oct nodes, then for each cell within that oct,
determining which cells must be selected or child nodes recursed into.  This
system is designed to allow for having leaf nodes of varying cells-per-side,
for instance 1, 2, 4, 8, etc.  However, the number of nodes is fixed at 8, with
subdivision always occurring at the midplane.

The final mechanism by which data is selected is for discrete data points,
typically particles in astrophysical simulations.  At present, this is done
by first identifying which data files intersect with a given selector, then
selecting individual points.  There is no hierarchical data selection conducted
in this system, as we do not yet allow for re-ordering of data on disk or
in-memory which would facilitate hierarchical selection through the use of
operations such as morton indices.

\textbf{add bitmap index stuff here}
