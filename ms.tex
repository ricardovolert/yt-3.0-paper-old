%\documentclass[12pt,preprint]{aastex}
\documentclass{emulateapj}

\newcommand{\Msun}{\ensuremath{\mbox{M}_\odot}}
\newcommand{\mhf}{\ensuremath{\mbox{f}_{\mbox{\tiny{H}}_2}}}
\newcommand{\ctr}{\ensuremath{c_{\mbox{\tiny{tr}}}}}
\newcommand{\cvib}{\ensuremath{c_{\mbox{\tiny{vib}}}}}
\newcommand{\crot}{\ensuremath{c_{\mbox{\tiny{rot}}}}}
\newcommand{\gcc}{\ensuremath{\mathrm{g}~\mathrm{cm}^{-3}}}
\newcommand{\pcc}{\ensuremath{\mathrm{amu}~\mathrm{cm}^{-3}}}
\newcommand{\cc}{\ensuremath{\mathrm{cm}^{-3}}}
\newcommand{\Hmol}{\ensuremath{\mathrm{H}_2}}
\newcommand{\msun}{\ensuremath{\mathrm{M}_\odot}}
\newcommand{\rarrow}{\ensuremath{\rightarrow}}
\newcommand{\cHI}{\ensuremath{\mathrm{H}}}
\newcommand{\cHII}{\ensuremath{\mathrm{H}^{+}}}
\newcommand{\cHM}{\ensuremath{\mathrm{H}^{-}}}
\newcommand{\cHeI}{\ensuremath{\mathrm{He}}}
\newcommand{\cHeII}{\ensuremath{\mathrm{He}^{+}}}
\newcommand{\cHeIII}{\ensuremath{\mathrm{He}^{++}}}
\newcommand{\celec}{\ensuremath{\mathrm{e}^{-}}}
\newcommand{\cHmolI}{\ensuremath{\mathrm{H}_{2}}}
\newcommand{\cHmolII}{\ensuremath{\mathrm{H}_{2}^{+}}}
\newcommand{\cDI}{\ensuremath{\mathrm{D}}}
\newcommand{\cDII}{\ensuremath{\mathrm{D}^{+}}}
\newcommand{\cHDI}{\ensuremath{\mathrm{HD}}}
\newcommand{\fH}{\ensuremath{f_{\mathrm{H}}}}
\newcommand{\fHmol}{\ensuremath{f_{\mathrm{H_2}}}}
\newcommand{\yt}{\texttt{yt}}

\bibliographystyle{apj}
\usepackage{natbib}
\usepackage{graphicx}
\usepackage{fancyvrb}
\fvset{fontsize=\footnotesize}


\begin{document}

\title{Introducing yt 3.0: Analysis and Visualization of Volumetric Data}
\author{The \yt{} Collaboration{1}}
\email{}
\altaffiltext{1}{The Globe} 

\begin{abstract}
\end{abstract}

\keywords{}

% Things that don't quite fit in the sections that absolutely bear mentioning:
%   * Stream
%   * All-sky
%   * Initial conditions generation

\maketitle

% n.b.:
% http://tex.stackexchange.com/questions/246/when-should-i-use-input-vs-include

\section{Introduction}
The process of transforming data into understanding constitutes the vast
majority of time, energy, and intellectual effort spent during scientific
inquiry.  This is true across domains, whether data is the product of a
computational simulation, a telescope observation, the synthesis of sensors
distributed across the Earth, or a collection of images of the human brain.
Data, by themselves, do not reflect an understanding of the Universe or its
underlying physical properties; rather, they are recordings, or measurements,
of the state of systems as observed.  Even for computational simulations, such
as simulations of star formation in the galaxy, this is true: these simulations
encode information about a discretization of a model, rather than the model
itself.

Bridging the gap between this discretization and the physical understanding
requires accessing data, manipulating and interrogating this data, and then
applying to this data a sense of understanding.  Somehow, bits stored on a disk
must become, in our minds, a galaxy undergoing a starburst.

This process is both mediated and impeded by computational tools.  When those
tools align with our mental model of how data exists, they can allow us to work
more efficiently, asking questions of data and building sophisticated
scientific inquiry.  However, when they do not, they can cause frustration,
delays, and most worryingly, incorrect or misinterpreted results.  When viewing
this from the perspective of the landscape of inquiry, the most startling
realization is that the questions a computational tool enables
individuals to ask shapes the questions they think to ask.

In \cite{yt_method_paper}, the analysis platform \yt{} was described.  At the
time, \yt{} was focused on analyzing and visualizing the output of grid-based
adaptive mesh refinement hydrodynamic simulations; while these were used to
study many different physical phenomena, they all were laid out in roughly the
same way, in rectilinear meshes of data.  In this paper, we present the current
version of \yt{}, which enables identical scripts to analyze and visualize data
stored as rectilinear grids as before, but additionally particle or discrete
data, octree-based data, and data stored as unstructured meshes.  This has been
the result of a large-scale effort to rewrite the underlying machinery within
\yt{} for accessing data, indexing that data, and providing it in efficient
ways to higher-level routines, as discussed in Section Something.  While this
was underway, \yt{} has also been considerably reinstrumented with
metadata-aware array infrastructure (Section \ref{sec:units}), the volume
rendering infrastructure has been rewritten to be more user-friendly and
capable (Section \ref{sec:vr}), and support for non-Cartesian geometries has
been added (Section \ref{sec:noncartesian}).

The single biggest change to \yt{} since that paper was published has not been
technical in nature.  In the intervening years, a directed and intense
community-building effort has resulted in the contributions from over a hundred
different individuals, many of them early-stage researchers, and a thriving
community of both users and developers (Section \ref{sec:community}).  This is
the crowning achievement of development, as we have attempted to build \yt{}
into a tool that enables inquiry from a technical level as well as fosters a
supportive, friendly community of individuals engaged in self-directed inquiry.


\section{Community Building}
% Things that need to be addressed here:
%
%  * How do we interact with the community of users
%  * How do we steward individuals to contribute and develop the code
%  * What is the governance procedure?
%  * How are priorities set (or how aren't they set)?


\section{Development Procedure}
\subsection{Answer Testing}

\subsection{Code Review}

\subsection{YTEP Process}




\section{Data Objects and Data Selection}
% Based on selectors
% Mention: Isocontours
The basic principles by which \yt{} operates are built on the notion of
selecting data (through coarse and subsequent fine-grained indexing of data
sources such as files), accessing that data in a memory-efficient fashion, and
then processing that data into either a resultant set of quantitative data or a
visualization.

The mechanisms by which \yt{} can select data are typically spatial in nature,
although several non-spatial mechanisms focused on queries can be utilized as
well.  These objects which conduct selection are selectors, and are designed to
provide as small of an API as possible, to enable ease of development and
deployment of new selectors.

Selectors require defining several functions, with the option of defining
additional functions for optimization, that return true or false whether a
given point is or is not included in the selected region.  These functions
include selection of a rectilinear grid (or any point within that grid),
selection of a point with zero extent and selection of a point with a non-zero
spherical radius.

The base selector object utilizes these routines during a selection operation
to maximize the amount of code reused between particle, patch, and octree
selection of data.  These three types of data are selected through specific
routines designed to minimize the number of times that the selection function
must be called, as they can be quite expensive.

Selecting data from a grid is a two-step process.  The first step is
identifying which grids intersect a given data selector; this is done through a
sequence of bounding box intersection checks.  Within a given grid, the cells
which are intersected are identified.  This results in the selection routine
being called once for each grid object in the simulation and once for each cell
located within an intersecting grid.  This can be conducted hierarchically, but
due to implementation details around how the grid index is stored this is not
yet cost effective.

Selecting data from an octree-organized dataset utilizes a recursive scheme
that selects individual oct nodes, then for each cell within that oct,
determining which cells must be selected or child nodes recursed into.  This
system is designed to allow for having leaf nodes of varying cells-per-side,
for instance 1, 2, 4, 8, etc.  However, the number of nodes is fixed at 8, with
subdivision always occurring at the midplane.

The final mechanism by which data is selected is for discrete data points,
typically particles in astrophysical simulations.  At present, this is done
by first identifying which data files intersect with a given selector, then
selecting individual points.  There is no hierarchical data selection conducted
in this system, as we do not yet allow for re-ordering of data on disk or
in-memory which would facilitate hierarchical selection through the use of
operations such as morton indices.

\textbf{add bitmap index stuff here}


\section{Data Processing and Analysis}
\input{sections/processing_and_analysis.tex}

\section{Visualization and Volume Rendering}
\input{sections/viz_and_volrendering.tex}

\section{Dimensional Analysis and Unit Systems}
\section{Dimensional Analysis and Units}

At a basic level, \texttt{yt} is an engine for converting data dumped to disk
by a simulation code into a physically meaningful result.  Attaching units to
simulation data makes it possible to perform dimensional analysis on the
simulation data, adding additional opportunities for catching errors in a data
processing pipeline.  In addition, it becomes straightforward to convert data
from one unit system to another.

In \texttt{yt 3.0} we handle units in a an automatic fashion, leveraging the
symbolic math library \texttt{sympy}. Instead of returning a NumPy
\texttt{ndarray} when users query \texttt{yt} data objects for fields, return a
\texttt{YTArray}, a subclass of \texttt{ndarray}. \texttt{YTArray} preserves
\texttt{ndarray}'s array operations, including deep and shallow copies,
broadcasting, and views.  Augmenting texttt{ndarray}, \texttt{YTArray} attaches
unit metadata to the array data, enabling runtime checking of unit consistency
in arithmetic operations between \texttt{YTArray} instances, and making it
trivial to compose new units using algebraic operations.

As a trivial example, when one queries a data object (here given the generic
name \texttt{dd}) for the density field, we get back a YTArray, including both
the simulation data for the density field, and the units of the density field,
in this case $\rm{g}/\rm{cm}^3$:

\begin{Verbatim}
>>> dd[`density'] 
YTArray([4.92e-31, 4.94e-31, 4.93e-31, ...,
         1.12e-25, 1.59e-25, 1.09e-24]) g/cm**3
\end{Verbatim}

One of the nicest aspects of this new unit system is that the symbolic
algebra for unitful operations is performed automatically by sympy:

\begin{Verbatim}
>>> print dd[`cell_mass']/dd[`cell_volume'] 
  [4.92e-31 4.94e-31 4.93e-31 ... 
   1.12e-25 1.59e-25 1.09e-24] g/cm**3
\end{Verbatim}

YTArray is primarily useful for attaching units to NumPy \texttt{ndarray}
instances. For scalar data, we have created the new \texttt{YTQuantity} class.
\texttt{YTQuantity} is a subclass of \texttt{YTArray} with the requirement that
the ``array data'' associated with the instance be limited to one
element. \texttt{YTQuantity} is primarily useful for physical constants and
ensures that the units are propogated correctly when composing quantities from
arrays, physical constants, and unitless scalars:

\begin{Verbatim}
>>> from yt.utilities.physical_constants import
        boltzmann_constant
>>> print dd[`temperature']*boltzmann_constant 
[ 1.28e-12 1.29e-12 1.29e-12 ... 
  1.63e-12 1.59e-12 1.40e-12] erg
\end{Verbatim}

If a user needs the field in a different unit system, they can quickly convert
using \texttt{convert\_to\_units} or \texttt{in\_units}.

When a \texttt{Dataset} object is instantiated, it will its self instantiate and
set up a \texttt{UnitRegistry} class that contains a full set of units that are
defined for the simulation. This registry includes both concrete physical units
like \texttt{cm} or \texttt{K} but also units symbols that correspond to the
unit system used internally in the simulation.

The new unit systems lets us to encode the simulation coordinate system
and scaling to physical coordinates directly into the unit system. We do
this via ``code units''.

Every \texttt{Dataset} has a \texttt{length\_unit},
\texttt{time\_unit}, and \texttt{mass\_unit},
attribute that the user can quickly and easily query to discover the
base units of the simulation. For example:

\begin{Verbatim}
>>> import yt
>>> ds = yt.load("Enzo_64/DD0043/data0043")
>>> print ds.length_unit 
128 Mpccm/h
>>> print ds.quan(1.0, "code_length").in_units("Mpccm/h")
128 Mpccm/h
>>> print ds.length_unit.in_cgs()
5.55517285026e+26 cm
\end{Verbatim}

Optionally \texttt{velocity\_unit}, \texttt{pressure\_unit},
\texttt{temperature\_unit}, and \texttt{density\_unit} may be defined as well if
the units for these fields cannot be inferred from the mass, length, and time
units.

Additionally, we allow conversions to the simulation unit system. Data in code
units are available by converting to \texttt{code\_length}, \texttt{code\_mass},
\texttt{code\_time}, \texttt{code\_velocity}, \texttt{code\_density},
\texttt{code\_magnetic}, \texttt{code\_pressure}, \texttt{code\_metallicity}, or
any combination of those units. Code units preserve dimensionality: an
array or quantity that has units of \texttt{cm} will be convertible to
\texttt{code\_length}, but not to \texttt{code\_mass}.

On-disk data are also be available to the user, presented in
unconverted code units. To obtain on-disk data, a user need only query a
data object using an on-disk field name:

\begin{Verbatim}
>>> import yt
>>> ds = yt.load(``Enzo\_64''/DD0043/data0043")
>>> dd = ds.all\_data()
>>> print dd[('enzo', 'Density')] 
[ 6.74e-02 6.12e-02 8.92e-02 ... 
  9.09e+01 5.66e+01 4.27e+01] code_mass/code_length**3
>>> print dd[('gas', 'density')] 
[ 1.92e-31 1.74e-31 2.54e-31 ... 
  2.59e-28 1.61e-28 1.22e-28] g/cm**3
\end{Verbatim}

Here, the first data object query is returned in code units, while the second is
returned in CGS units. This is because \verb!(`enzo', `Density')! is an on-disk
field, while \verb!(`gas', `density')! is an internal \texttt{yt} field.

\subsubsection{Implementation}\label{implementation}

Our unit system has 6 base dimensions, \texttt{mass}, \texttt{length},
\texttt{time}, \texttt{temperature}, and \texttt{angle}. The unitless
\texttt{dimensionless} dimension, which we use to represent scalars is also
technically a base dimension, although a trivial one.  For convenience, we also
create dimensionless unit symbols to represent quantities like metallicity that
are formally dimensionless, but it is convenient to represent in a unit system.

For each dimension, we choose a base unit. Our system's base units are grams,
centimeters, seconds, Kelvin, and radian.  All units can be described as
combinations of these base dimensions along with a conversion factor to
equivalent base units.

The choice of CGS as the base unit system is somewhat arbitrary. Most unit
systems choose SI as the reference unit system. We use CGS to stay consistent
with the rest of the \texttt{yt} codebase and to reflect the standard practice
in astrophysics. In any case, using a \emph{physical} coordinate system makes it
possible to compare quantities and arrays produced by different datasets,
possibly with different conversion factors to CGS and to code units. We go into
more detail on this point below.  In the future, we plan to make the preferred
internal coordinate system a user-configurable option.

We provide sympy \texttt{Symbol} objects for the base dimensions. The
dimensionality of all other units should be \texttt{sympy} \texttt{Expr} objects
made up of the base dimension objects and the \texttt{sympy} operation objects
\texttt{Mul} and \texttt{Pow}.

Let's use some common units as examples: gram (\texttt{g}), erg (\texttt{erg}),
and solar mass per cubic megaparsec (\texttt{Msun / Mpc$^3$}). \texttt{g} is an
atomic, CGS base unit, \texttt{erg} is an atomic unit in CGS, but is not a base
unit, and \texttt{Msun/Mpc$^3$} is a combination of atomic units, which are not
in CGS, and one of them even has an SI prefix. The dimensions of \texttt{g} are
\texttt{mass} and the cgs factor is 1. The dimensions of \texttt{erg} are
\texttt{mass * length$^2$ * time$^{-2}$} and the cgs factor is 1. The dimensions
of \texttt{Msun/Mpc$^3$} are \texttt{mass / length$^3$} and the cgs factor is
about 6.8e-41.

We use the \texttt{UnitRegistry} class to define all valid atomic units.
All unit registries contain a unit symbol lookup table (dict) containing
the valid units' dimensionality and cgs conversion factor. Here is what
it would look like with the above units:

\begin{Verbatim}
{ "g":    (mass, 1.0),
  "erg":  (mass * length**2 * time**-2, 1.0),
  "Msun": (mass, 1.98892e+33),
  "pc":   (length, 3.08568e18), }
\end{Verbatim}

Note that we only define \emph{atomic} units here. There should be no
operations in the registry symbol strings. When we parse non-atomic
units like \texttt{Msun/Mpc**3}, we use the registry to look up the
symbols. The unit system in yt knows how to handle units like
\texttt{Mpc} by looking up unit symbols with and without prefixes and
modify the conversion factor appropriately.

We construct a \texttt{Unit} object by providing a string containing
atomic unit symbols, combined with operations in Python syntax, and the
registry those atomic unit symbols are defined in. We use sympy's string
parsing features to create the unit expression from the user-provided
string.

\texttt{Unit} objects are associated with four instance members, a unit
\texttt{Expression} object, a dimensionality \texttt{Expression} object,
a \texttt{UnitRegistry} instance, and a scalar conversion factor to CGS
units. These data are available for a \texttt{Unit} object by accessing
the \texttt{expr}, \texttt{dimensions}, \texttt{registry}, and
\texttt{cgs\_value} attributes, respectively.

\texttt{Unit} provides the methods \texttt{same\_dimensions\_as}, which
returns True if passed a \texttt{Unit} object that has equivalent
dimensions, \texttt{get\_cgs\_equivalent}, which returns the equivalent
cgs base units of the \texttt{Unit}, and the \texttt{is\_code\_unit}
property, which is \texttt{True} if the unit is composed purely of code
units and \texttt{False} otherwise. \texttt{Unit} also defines the
\texttt{mul}, \texttt{div}, \texttt{pow}, and \texttt{eq} operations
with other unit objects, making it easy to compose compound units
algebraically.

The \texttt{UnitRegistry} class provides the \texttt{add},
\texttt{remove}, and \texttt{modify} methods which allows users to add,
remove, and modify atomic unit definitions present in
\texttt{UnitRegistry} objects. A dictionary lookup table is also
attached to the \texttt{UnitRegistry} object, providing an interface to
look up unit symbols. In general, unit registries should only be
adjusted inside of a code frontend, since otherwise quantities and
arrays might be created with inconsistent unit metadata. Once a unit
object is created, it will not recieve updates if the original unit
registry is modified.

\subsubsection{Creating YTArray and YTQuantity
instances}\label{creating-ytarray-and-ytquantity-instances}

There are two ways to create new array and quantity objects: via a constructor,
and by multiplying scalar data by a unit quantity.

\paragraph{Class Constructor}\label{class-constructor}

The primary internal interface for creating new arrays and quantities is through
the class constructor for YTArray. The constructor takes three arguments. The
first argument is the input scalar data, which can be an integer, float, list,
or array. The second argument, \texttt{input\_units}, is a unit specification
which must be a string or \texttt{Unit} instance. Last, users may optionally
supply a UnitRegistry instance, which will be attached to the array. If no
UnitRegistry is supplied, a default unit registry is used instead. Unit
specification strings must be algebraic combinations of unit symbol names, using
standard Python mathematical syntax (i.e. \texttt{**} for the power function,
not \texttt{\^{}}).

Here is a simple example of \texttt{YTArray} creation:

\begin{Verbatim}
>>> from yt.units import yt\_array, YTQuantity 
>>> YTArray([1, 2, 3], `cm') 
YTArray([1, 2, 3]) cm
>>> YTQuantity(3, `J') 
3 J
\end{Verbatim}

In addition to the class constructor, we have also defined two
convenience functions, \texttt{quan}, and \texttt{arr}, for quantity and
array creation that are attached to the \texttt{Dataset} base
class. These were added to syntactically simplify the creation of arrays
with the UnitRegistry instance associated with a dataset. These
functions work exactly like the \texttt{YTArray} and \texttt{YTQuantity}
constructors, but pass the \texttt{UnitRegistry} instance attached to
the dataset to the underlying constructor call. For example:

\begin{Verbatim}
>>> import yt
>>> ds = yt.load(``Enzo\_64/DD0043/data0043'')
>>> ds.arr([1, 2, 3], `code_length').in_cgs() 
YTArray([ 5.55e+26, 1.11e+27, 1.66e+27]) cm
\end{Verbatim}

This example illustrates that the array is being created using
\texttt{ds.unit\_registry}, rather than the
\texttt{default\_unit\_registry}, for which \texttt{code\_length} is
equivalent to \texttt{cm}.

\paragraph{Multiplication}\label{multiplication}

New \texttt{YTArray} and \texttt{YTQuantity} instances can also be created by
multiplying \texttt{YTArray} or \texttt{YTQuantity} instances by \texttt{float}
or \texttt{ndarray} instances. To make it easier to create arrays using this
mechanism, we have populated the \texttt{yt.units} namespace with predefined
\texttt{YTQuantity} instances that correspond to common unit symbol names. For
example:

\begin{Verbatim}
>>> from yt.units import meter, gram, kilogram, second, joule 
>>> kilogram * meter**2 == joule 
True
>>> from yt.units import m, kg, s, W 
>>> kg*m**2/s**3 == W
True

>>> from yt.units import kilometer 
>>> three_kilometers = 3*kilometer 
>>> print three_kilometers 
3.0 km

>>> from yt.units import gram, kilogram 
>>> print gram+kilogram 
1001.0 g 
>>> print kilogram+gram 
1.001 kg 
>>> print kilogram/gram 
1000.0 dimensionless
\end{Verbatim}

\subsubsection{Handling code units}\label{handling-code-units}


Code units are tightly coupled to on-disk parameters. To handle this
fact of life, the \texttt{yt} unit system can modify, add, and remove
unit symbols via the \texttt{UnitRegistry}.

\paragraph{Associating arrays with a coordinate
system}\label{associating-arrays-with-a-coordinate-system}

To create quantities and arrays in units defined by a simulation
coordinate system, we associate a \texttt{UnitRegistry} instance with
\texttt{Dataset} instances. This unit registry contains the
metadata necessary to convert the array to CGS from some other known
unit system and is available via the \texttt{unit\_registry} attribute
that is attached to all \texttt{Dataset} instances.

We have modified the definition for \texttt{set\_code\_units} in the
\texttt{StaticOutput} base class. In this new implemenation, the
predefined \texttt{code\_mass}, \texttt{code\_length},
\texttt{code\_time}, and \texttt{code\_velocity} symbols are adjusted to
the appropriate values and \texttt{length\_unit}, \texttt{time\_unit},
\texttt{mass\_unit}, \texttt{velocity\_unit} attributes are attached to
the \texttt{StaticOutput} instance. If there are frontend specific code
units they should also be defined in subclasses by extending this function.

\paragraph{Mixing modified unit
registries}\label{mixing-modified-unit-registries}

It becomes necessary to consider mixing unit registries whenever data
needs to be compared between disparate datasets. The most
straightforward example where this comes up is a cosmological simulation
time series, where the code units evolve with time. The problem is quite
general --- we want to be able to compare any two datasets, even if they
are unrelated.

We have designed the unit system to refer to a physical coordinate
system based on CGS conversion factors. This means that operations on
quantities with different unit registries will always agree since the
final calculation is always performed in CGS.

The examples below illustrate the consistency of this choice:

\begin{Verbatim}
>>> import yt
>>> ds1 = yt.load(`Enzo_64/DD0002/data0002')
>>> ds2 = yt.load(`Enzo_64/DD0043/data0043')
>>> print ds1.length_unit, ds2.length_unit 
128 Mpccm/h, 128 Mpccm/h
>>> print ds1.length_unit.in_cgs()
6.26145538088e+25 cm
>>> print ds2.length_unit.in_cgs() 
5.55517285026e+26 cm 


>>> print ds1.length_unit*ds2.length_unit 
145359.100149 Mpccm**2
>>> print ds2.length_unit*ds1.length_unit 
1846.7055432 Mpccm**2
\end{Verbatim}

For the last two examples, the answer is not the seemingly trivial
$128^2\/=\/16384\,\rm{Mpccm}^2/h^2$. This is because the new quantity
returned by the multiplication operation inherits the unit registry from
the left object in binary operations. This convention is enforced for
all binary operations on two \texttt{YTarray} objects. Results are
always consistent when referencing an unambiguous physical coordinate system:

\begin{Verbatim}
>>> print (pf1.length_unit * pf2.length_unit).in_cgs() 
3.4783466935e+52 cm**2 
>>> print pf1.length_unit.in_cgs() * pf2.length_unit.in_cgs() 
3.4783466935e+52 cm**2
\end{Verbatim}

\subsubsection{Handling cosmological
units}\label{handling-cosmological-units}

If we detect that we are loading a cosmological simulation performed in comoving
coordinates, extra comoving units are added to the dataset's unit
registry. Comoving length unit symbols are still named following the pattern
\texttt{<length symbol>cm}, i.e. \texttt{Mpccm}.

The $h$ symbol is treated as a base unit, \texttt{h}, which defaults to
unity. The \texttt{Dataset.set\_units} updates the \texttt{h} symbol to
the correct value when loading a cosmological simulation.


\section{Abstracting Simulation Types}
\subsection{Decomposition Strategies}

\subsection{Grid Analysis}

\subsection{Octree Analysis}

\subsection{SPH Analysis}

\subsection{Unstructured Mesh Analysis}

\subsection{Non-Cartesian Coordinates}




\section{Units and Quantities}
\section{Dimensional Analysis and Units}

At a basic level, \texttt{yt} is an engine for converting data dumped to disk
by a simulation code into a physically meaningful result.  Attaching units to
simulation data makes it possible to perform dimensional analysis on the
simulation data, adding additional opportunities for catching errors in a data
processing pipeline.  In addition, it becomes straightforward to convert data
from one unit system to another.

In \texttt{yt 3.0} we handle units in a an automatic fashion, leveraging the
symbolic math library \texttt{sympy}. Instead of returning a NumPy
\texttt{ndarray} when users query \texttt{yt} data objects for fields, return a
\texttt{YTArray}, a subclass of \texttt{ndarray}. \texttt{YTArray} preserves
\texttt{ndarray}'s array operations, including deep and shallow copies,
broadcasting, and views.  Augmenting texttt{ndarray}, \texttt{YTArray} attaches
unit metadata to the array data, enabling runtime checking of unit consistency
in arithmetic operations between \texttt{YTArray} instances, and making it
trivial to compose new units using algebraic operations.

As a trivial example, when one queries a data object (here given the generic
name \texttt{dd}) for the density field, we get back a YTArray, including both
the simulation data for the density field, and the units of the density field,
in this case $\rm{g}/\rm{cm}^3$:

\begin{Verbatim}
>>> dd[`density'] 
YTArray([4.92e-31, 4.94e-31, 4.93e-31, ...,
         1.12e-25, 1.59e-25, 1.09e-24]) g/cm**3
\end{Verbatim}

One of the nicest aspects of this new unit system is that the symbolic
algebra for unitful operations is performed automatically by sympy:

\begin{Verbatim}
>>> print dd[`cell_mass']/dd[`cell_volume'] 
  [4.92e-31 4.94e-31 4.93e-31 ... 
   1.12e-25 1.59e-25 1.09e-24] g/cm**3
\end{Verbatim}

YTArray is primarily useful for attaching units to NumPy \texttt{ndarray}
instances. For scalar data, we have created the new \texttt{YTQuantity} class.
\texttt{YTQuantity} is a subclass of \texttt{YTArray} with the requirement that
the ``array data'' associated with the instance be limited to one
element. \texttt{YTQuantity} is primarily useful for physical constants and
ensures that the units are propogated correctly when composing quantities from
arrays, physical constants, and unitless scalars:

\begin{Verbatim}
>>> from yt.utilities.physical_constants import
        boltzmann_constant
>>> print dd[`temperature']*boltzmann_constant 
[ 1.28e-12 1.29e-12 1.29e-12 ... 
  1.63e-12 1.59e-12 1.40e-12] erg
\end{Verbatim}

If a user needs the field in a different unit system, they can quickly convert
using \texttt{convert\_to\_units} or \texttt{in\_units}.

When a \texttt{Dataset} object is instantiated, it will its self instantiate and
set up a \texttt{UnitRegistry} class that contains a full set of units that are
defined for the simulation. This registry includes both concrete physical units
like \texttt{cm} or \texttt{K} but also units symbols that correspond to the
unit system used internally in the simulation.

The new unit systems lets us to encode the simulation coordinate system
and scaling to physical coordinates directly into the unit system. We do
this via ``code units''.

Every \texttt{Dataset} has a \texttt{length\_unit},
\texttt{time\_unit}, and \texttt{mass\_unit},
attribute that the user can quickly and easily query to discover the
base units of the simulation. For example:

\begin{Verbatim}
>>> import yt
>>> ds = yt.load("Enzo_64/DD0043/data0043")
>>> print ds.length_unit 
128 Mpccm/h
>>> print ds.quan(1.0, "code_length").in_units("Mpccm/h")
128 Mpccm/h
>>> print ds.length_unit.in_cgs()
5.55517285026e+26 cm
\end{Verbatim}

Optionally \texttt{velocity\_unit}, \texttt{pressure\_unit},
\texttt{temperature\_unit}, and \texttt{density\_unit} may be defined as well if
the units for these fields cannot be inferred from the mass, length, and time
units.

Additionally, we allow conversions to the simulation unit system. Data in code
units are available by converting to \texttt{code\_length}, \texttt{code\_mass},
\texttt{code\_time}, \texttt{code\_velocity}, \texttt{code\_density},
\texttt{code\_magnetic}, \texttt{code\_pressure}, \texttt{code\_metallicity}, or
any combination of those units. Code units preserve dimensionality: an
array or quantity that has units of \texttt{cm} will be convertible to
\texttt{code\_length}, but not to \texttt{code\_mass}.

On-disk data are also be available to the user, presented in
unconverted code units. To obtain on-disk data, a user need only query a
data object using an on-disk field name:

\begin{Verbatim}
>>> import yt
>>> ds = yt.load(``Enzo\_64''/DD0043/data0043")
>>> dd = ds.all\_data()
>>> print dd[('enzo', 'Density')] 
[ 6.74e-02 6.12e-02 8.92e-02 ... 
  9.09e+01 5.66e+01 4.27e+01] code_mass/code_length**3
>>> print dd[('gas', 'density')] 
[ 1.92e-31 1.74e-31 2.54e-31 ... 
  2.59e-28 1.61e-28 1.22e-28] g/cm**3
\end{Verbatim}

Here, the first data object query is returned in code units, while the second is
returned in CGS units. This is because \verb!(`enzo', `Density')! is an on-disk
field, while \verb!(`gas', `density')! is an internal \texttt{yt} field.

\subsubsection{Implementation}\label{implementation}

Our unit system has 6 base dimensions, \texttt{mass}, \texttt{length},
\texttt{time}, \texttt{temperature}, and \texttt{angle}. The unitless
\texttt{dimensionless} dimension, which we use to represent scalars is also
technically a base dimension, although a trivial one.  For convenience, we also
create dimensionless unit symbols to represent quantities like metallicity that
are formally dimensionless, but it is convenient to represent in a unit system.

For each dimension, we choose a base unit. Our system's base units are grams,
centimeters, seconds, Kelvin, and radian.  All units can be described as
combinations of these base dimensions along with a conversion factor to
equivalent base units.

The choice of CGS as the base unit system is somewhat arbitrary. Most unit
systems choose SI as the reference unit system. We use CGS to stay consistent
with the rest of the \texttt{yt} codebase and to reflect the standard practice
in astrophysics. In any case, using a \emph{physical} coordinate system makes it
possible to compare quantities and arrays produced by different datasets,
possibly with different conversion factors to CGS and to code units. We go into
more detail on this point below.  In the future, we plan to make the preferred
internal coordinate system a user-configurable option.

We provide sympy \texttt{Symbol} objects for the base dimensions. The
dimensionality of all other units should be \texttt{sympy} \texttt{Expr} objects
made up of the base dimension objects and the \texttt{sympy} operation objects
\texttt{Mul} and \texttt{Pow}.

Let's use some common units as examples: gram (\texttt{g}), erg (\texttt{erg}),
and solar mass per cubic megaparsec (\texttt{Msun / Mpc$^3$}). \texttt{g} is an
atomic, CGS base unit, \texttt{erg} is an atomic unit in CGS, but is not a base
unit, and \texttt{Msun/Mpc$^3$} is a combination of atomic units, which are not
in CGS, and one of them even has an SI prefix. The dimensions of \texttt{g} are
\texttt{mass} and the cgs factor is 1. The dimensions of \texttt{erg} are
\texttt{mass * length$^2$ * time$^{-2}$} and the cgs factor is 1. The dimensions
of \texttt{Msun/Mpc$^3$} are \texttt{mass / length$^3$} and the cgs factor is
about 6.8e-41.

We use the \texttt{UnitRegistry} class to define all valid atomic units.
All unit registries contain a unit symbol lookup table (dict) containing
the valid units' dimensionality and cgs conversion factor. Here is what
it would look like with the above units:

\begin{Verbatim}
{ "g":    (mass, 1.0),
  "erg":  (mass * length**2 * time**-2, 1.0),
  "Msun": (mass, 1.98892e+33),
  "pc":   (length, 3.08568e18), }
\end{Verbatim}

Note that we only define \emph{atomic} units here. There should be no
operations in the registry symbol strings. When we parse non-atomic
units like \texttt{Msun/Mpc**3}, we use the registry to look up the
symbols. The unit system in yt knows how to handle units like
\texttt{Mpc} by looking up unit symbols with and without prefixes and
modify the conversion factor appropriately.

We construct a \texttt{Unit} object by providing a string containing
atomic unit symbols, combined with operations in Python syntax, and the
registry those atomic unit symbols are defined in. We use sympy's string
parsing features to create the unit expression from the user-provided
string.

\texttt{Unit} objects are associated with four instance members, a unit
\texttt{Expression} object, a dimensionality \texttt{Expression} object,
a \texttt{UnitRegistry} instance, and a scalar conversion factor to CGS
units. These data are available for a \texttt{Unit} object by accessing
the \texttt{expr}, \texttt{dimensions}, \texttt{registry}, and
\texttt{cgs\_value} attributes, respectively.

\texttt{Unit} provides the methods \texttt{same\_dimensions\_as}, which
returns True if passed a \texttt{Unit} object that has equivalent
dimensions, \texttt{get\_cgs\_equivalent}, which returns the equivalent
cgs base units of the \texttt{Unit}, and the \texttt{is\_code\_unit}
property, which is \texttt{True} if the unit is composed purely of code
units and \texttt{False} otherwise. \texttt{Unit} also defines the
\texttt{mul}, \texttt{div}, \texttt{pow}, and \texttt{eq} operations
with other unit objects, making it easy to compose compound units
algebraically.

The \texttt{UnitRegistry} class provides the \texttt{add},
\texttt{remove}, and \texttt{modify} methods which allows users to add,
remove, and modify atomic unit definitions present in
\texttt{UnitRegistry} objects. A dictionary lookup table is also
attached to the \texttt{UnitRegistry} object, providing an interface to
look up unit symbols. In general, unit registries should only be
adjusted inside of a code frontend, since otherwise quantities and
arrays might be created with inconsistent unit metadata. Once a unit
object is created, it will not recieve updates if the original unit
registry is modified.

\subsubsection{Creating YTArray and YTQuantity
instances}\label{creating-ytarray-and-ytquantity-instances}

There are two ways to create new array and quantity objects: via a constructor,
and by multiplying scalar data by a unit quantity.

\paragraph{Class Constructor}\label{class-constructor}

The primary internal interface for creating new arrays and quantities is through
the class constructor for YTArray. The constructor takes three arguments. The
first argument is the input scalar data, which can be an integer, float, list,
or array. The second argument, \texttt{input\_units}, is a unit specification
which must be a string or \texttt{Unit} instance. Last, users may optionally
supply a UnitRegistry instance, which will be attached to the array. If no
UnitRegistry is supplied, a default unit registry is used instead. Unit
specification strings must be algebraic combinations of unit symbol names, using
standard Python mathematical syntax (i.e. \texttt{**} for the power function,
not \texttt{\^{}}).

Here is a simple example of \texttt{YTArray} creation:

\begin{Verbatim}
>>> from yt.units import yt\_array, YTQuantity 
>>> YTArray([1, 2, 3], `cm') 
YTArray([1, 2, 3]) cm
>>> YTQuantity(3, `J') 
3 J
\end{Verbatim}

In addition to the class constructor, we have also defined two
convenience functions, \texttt{quan}, and \texttt{arr}, for quantity and
array creation that are attached to the \texttt{Dataset} base
class. These were added to syntactically simplify the creation of arrays
with the UnitRegistry instance associated with a dataset. These
functions work exactly like the \texttt{YTArray} and \texttt{YTQuantity}
constructors, but pass the \texttt{UnitRegistry} instance attached to
the dataset to the underlying constructor call. For example:

\begin{Verbatim}
>>> import yt
>>> ds = yt.load(``Enzo\_64/DD0043/data0043'')
>>> ds.arr([1, 2, 3], `code_length').in_cgs() 
YTArray([ 5.55e+26, 1.11e+27, 1.66e+27]) cm
\end{Verbatim}

This example illustrates that the array is being created using
\texttt{ds.unit\_registry}, rather than the
\texttt{default\_unit\_registry}, for which \texttt{code\_length} is
equivalent to \texttt{cm}.

\paragraph{Multiplication}\label{multiplication}

New \texttt{YTArray} and \texttt{YTQuantity} instances can also be created by
multiplying \texttt{YTArray} or \texttt{YTQuantity} instances by \texttt{float}
or \texttt{ndarray} instances. To make it easier to create arrays using this
mechanism, we have populated the \texttt{yt.units} namespace with predefined
\texttt{YTQuantity} instances that correspond to common unit symbol names. For
example:

\begin{Verbatim}
>>> from yt.units import meter, gram, kilogram, second, joule 
>>> kilogram * meter**2 == joule 
True
>>> from yt.units import m, kg, s, W 
>>> kg*m**2/s**3 == W
True

>>> from yt.units import kilometer 
>>> three_kilometers = 3*kilometer 
>>> print three_kilometers 
3.0 km

>>> from yt.units import gram, kilogram 
>>> print gram+kilogram 
1001.0 g 
>>> print kilogram+gram 
1.001 kg 
>>> print kilogram/gram 
1000.0 dimensionless
\end{Verbatim}

\subsubsection{Handling code units}\label{handling-code-units}


Code units are tightly coupled to on-disk parameters. To handle this
fact of life, the \texttt{yt} unit system can modify, add, and remove
unit symbols via the \texttt{UnitRegistry}.

\paragraph{Associating arrays with a coordinate
system}\label{associating-arrays-with-a-coordinate-system}

To create quantities and arrays in units defined by a simulation
coordinate system, we associate a \texttt{UnitRegistry} instance with
\texttt{Dataset} instances. This unit registry contains the
metadata necessary to convert the array to CGS from some other known
unit system and is available via the \texttt{unit\_registry} attribute
that is attached to all \texttt{Dataset} instances.

We have modified the definition for \texttt{set\_code\_units} in the
\texttt{StaticOutput} base class. In this new implemenation, the
predefined \texttt{code\_mass}, \texttt{code\_length},
\texttt{code\_time}, and \texttt{code\_velocity} symbols are adjusted to
the appropriate values and \texttt{length\_unit}, \texttt{time\_unit},
\texttt{mass\_unit}, \texttt{velocity\_unit} attributes are attached to
the \texttt{StaticOutput} instance. If there are frontend specific code
units they should also be defined in subclasses by extending this function.

\paragraph{Mixing modified unit
registries}\label{mixing-modified-unit-registries}

It becomes necessary to consider mixing unit registries whenever data
needs to be compared between disparate datasets. The most
straightforward example where this comes up is a cosmological simulation
time series, where the code units evolve with time. The problem is quite
general --- we want to be able to compare any two datasets, even if they
are unrelated.

We have designed the unit system to refer to a physical coordinate
system based on CGS conversion factors. This means that operations on
quantities with different unit registries will always agree since the
final calculation is always performed in CGS.

The examples below illustrate the consistency of this choice:

\begin{Verbatim}
>>> import yt
>>> ds1 = yt.load(`Enzo_64/DD0002/data0002')
>>> ds2 = yt.load(`Enzo_64/DD0043/data0043')
>>> print ds1.length_unit, ds2.length_unit 
128 Mpccm/h, 128 Mpccm/h
>>> print ds1.length_unit.in_cgs()
6.26145538088e+25 cm
>>> print ds2.length_unit.in_cgs() 
5.55517285026e+26 cm 


>>> print ds1.length_unit*ds2.length_unit 
145359.100149 Mpccm**2
>>> print ds2.length_unit*ds1.length_unit 
1846.7055432 Mpccm**2
\end{Verbatim}

For the last two examples, the answer is not the seemingly trivial
$128^2\/=\/16384\,\rm{Mpccm}^2/h^2$. This is because the new quantity
returned by the multiplication operation inherits the unit registry from
the left object in binary operations. This convention is enforced for
all binary operations on two \texttt{YTarray} objects. Results are
always consistent when referencing an unambiguous physical coordinate system:

\begin{Verbatim}
>>> print (pf1.length_unit * pf2.length_unit).in_cgs() 
3.4783466935e+52 cm**2 
>>> print pf1.length_unit.in_cgs() * pf2.length_unit.in_cgs() 
3.4783466935e+52 cm**2
\end{Verbatim}

\subsubsection{Handling cosmological
units}\label{handling-cosmological-units}

If we detect that we are loading a cosmological simulation performed in comoving
coordinates, extra comoving units are added to the dataset's unit
registry. Comoving length unit symbols are still named following the pattern
\texttt{<length symbol>cm}, i.e. \texttt{Mpccm}.

The $h$ symbol is treated as a base unit, \texttt{h}, which defaults to
unity. The \texttt{Dataset.set\_units} updates the \texttt{h} symbol to
the correct value when loading a cosmological simulation.


\section{User-Friendliness}
\subsection{Publication-Ready Figures}

\subsection{IPython Integration and HTML GUIs}




\section{Halo Finding and Catalogs}
\input{sections/halo_finding_and_catalogs.tex}

\section{Scaling and Parallelism}
\subsection{Performance of Operations}
% Profiles
% Quadtree

\subsection{Inline Analysis}

\subsection{Simple Parallelism}
% Parallel objects




\section{Simulation Support Structures}
\subsection{GDF-IO}

\subsection{Initial Conditions Generation}




\section{Astrophysical Analysis Modules}
\input{sections/analysis_modules.tex}
% Tons of these.
% Mention external packages we interface with

\section{Future Directions}
\input{sections/future_directions.tex}

\section{Conclusions}\label{sec:conclusions}
\input{sections/conclusions.tex}

\acknowledgments 
\section{Acknowledgments}

The authors of this paper would like to extend their deepest gratitude to the
many, many individual and institutions that have contributed, directly or
indirectly, to the growth of both \yt{} and the \yt{} community.

% institutions that have provided considerable support

We
particularly thank KIPAC and SLAC at Stanford, the University of California at
San Diego and Santa Cruz, the High-Performance Astro Computing Center, Columbia
University, the University of Illinois, University of Colorado at Boulder,
University of Edinburgh, the scientific python community, NumFOCUS, 

% grants here


% ADS library:
% http://adsabs.harvard.edu/cgi-bin/nph-abs_connect?library&libname=yt&libid=54a19bc904
\bibliography{mjt}

\end{document}
